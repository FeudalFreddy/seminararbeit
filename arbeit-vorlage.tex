\documentclass[fontsize=12pt, paper=a4, headinclude, twoside=false, parskip=half+, pagesize=auto, numbers=noenddot, plainheadsepline, open=right, toc=listof, toc=bibliography]{scrreprt}
% PDF-Kompression
\pdfminorversion=5
\pdfobjcompresslevel=1

%Versuche
\usepackage{wallpaper}



% Allgemeines
\usepackage[automark,plainfootsepline,headsepline]{scrlayer-scrpage} % Kopf- und Fußzeilen
\usepackage{todonotes}
\usepackage{color}
\usepackage{ulem}


%Versuch das Seitenlayout zu ändern
\usepackage{layout}
%\setlength{\hoffset}{18mm}
%\setlength{\textwidth}{145mm}

%\geometry{a4paper, total={210mm, 297mm}, left=30mm, right=20m, top=25mm, bottom=25mm}
%\voffset=-5mm
%\hoffset=18mm
%\marginparwidth15mm
%\textwidth145mm
%\textheight300mm

	
\usepackage{amsmath,marvosym} % Mathesachen
\usepackage{mathtools} % Mathesachen 
\usepackage{amstext}
\usepackage[T1]{fontenc} %Ligaturen, richtige Umlaute im PDF
\usepackage[utf8]{inputenc} %UTF8-Kodierung für Umlaute usw

%\usepackage[applemac]{inputenc} %Wenn Sie ein Mac-Nutzer sind, dann nutzen Sie dieses Paket um Umlaute für Mac korrekt wiederzugeben. Aber Sie müssen dann das UTF8 Paket deaktivieren.

%\usepackage[latin1]{inputenc} %Falls Sie Linux verwende

%Versuch
\renewcommand*\familydefault{\sfdefault}
%\usepackage{dejavu}
\usepackage[scaled]{berasans}

% Schriften
\usepackage[printonlyused]{acronym} %Notwendig für Abkürzungsverzeichnis
\usepackage{mathpazo} % Palatino für Mathemodus
\usepackage{setspace} % Zeilenabstand
\onehalfspacing % 1,5 Zeilen

% Schriften-Größen
\setkomafont{chapter}{\Large\sffamily} % Überschrift der Ebene
\renewcommand*{\chapterheadstartvskip}{\vspace*{-1mm}}
\setkomafont{section}{\large\sffamily}
\setkomafont{subsection}{\normalsize\sffamily}
\setkomafont{subsubsection}{\normalsize\sffamily}
\setkomafont{paragraph}{\normalsize\sffamily}
\setkomafont{subparagraph}{\normalsize\sffamily}
\setkomafont{chapterentry}{\normalsize\sffamily} % Überschrift der Ebene in Inhaltsverzeichnis
\setkomafont{descriptionlabel}{\bfseries\sffamily} % für description Umgebungen
\setkomafont{captionlabel}{\small\bfseries}
\setkomafont{caption}{\small}

\setkomafont{pageheadfoot}{\normalfont}

%Versuch: Überschriften Abstände zu verkleinern
\RedeclareSectionCommand[%
  beforeskip=-0.2\baselineskip,
  afterskip=0.2\baselineskip]{chapter}
\RedeclareSectionCommand[%
  beforeskip=-0.2\baselineskip,
  afterskip=0.2\baselineskip]{section}
\RedeclareSectionCommand[
  beforeskip=-.2\baselineskip,
  afterskip=.2\baselineskip]{subsection}
\RedeclareSectionCommand[
  beforeskip=-.15\baselineskip,
  afterskip=.15\baselineskip]{subsubsection}
\RedeclareSectionCommand[
  beforeskip=.15\baselineskip,
  afterskip=-1em]{paragraph}
\RedeclareSectionCommand[
  beforeskip=-.15\baselineskip,
  afterskip=-1em]{subparagraph}

% Sprache: Deutsch
\usepackage[ngerman]{babel} % Silbentrennung
% PDF
\usepackage[ngerman,pdfauthor={Frederik Hendricks-Kühn},  pdfauthor={Frederik Hendricks-Kühn}, pdftitle={Seminararbeit}, breaklinks=true,baseurl={}]{hyperref}
\usepackage[final]{microtype} % mikrotypographische Optimierungen
\usepackage{url}
\usepackage{pdflscape} % einzelne Seiten drehen können
% Tabellen
\usepackage{multirow} % Tabellen-Zellen über mehrere Zeilen
\usepackage{multicol} % mehre Spalten auf eine Seite
\usepackage{tabularx} % Für Tabellen mit vorgegeben Größen
\usepackage{longtable} % Tabellen über mehrere Seiten
\usepackage{booktabs} % schoenere Tabellen + abstaende
\usepackage{array}

\newcolumntype{L}[1]{>{\raggedright\arraybackslash}p{#1}} % linksbündig mit Breitenangabe
\newcolumntype{C}[1]{>{\centering\arraybackslash}p{#1}} % zentriert mit Breitenangabe
\newcolumntype{R}[1]{>{\raggedleft\arraybackslash}p{#1}} % rechtsbündig mit Breitenangabe


%Versuch die Schriftgröße in den Tabellen zu ändern
\let\oldtabular\tabular 
\renewcommand{\tabular}{\footnotesize\oldtabular}

%  Bibliographie
\usepackage{bibgerm} % Umlaute in BibTeX
% Tabellen
\usepackage{multirow} % Tabellen-Zellen über mehrere Zeilen
\usepackage{multicol} % mehre Spalten auf eine Seite
\usepackage{tabularx} % Für Tabellen mit vorgegeben Größen
\usepackage{array}
\usepackage{float}
\usepackage{titling}
\usepackage{enumitem} 
% Bilder
\usepackage{graphicx} % Bilder
\usepackage{color} % Farben
\graphicspath{{images/}}
\DeclareGraphicsExtensions{.pdf,.png,.jpg} % bevorzuge pdf-Dateien
\usepackage{subfig}
\usepackage[subfigure, titles]{tocloft}
\renewcommand{\cftchapdotsep}{\cftdotsep}
\renewcommand{\cftchapleader}{\cftdotfill{\cftchapdotsep}}
\newcommand{\subfigureautorefname}{\figurename} % um \autoref auch für subfigures benutzen
\usepackage[all]{hypcap} % Beim Klicken auf Links zum Bild und nicht zu Caption gehen
% Bildunterschrift
\setcapindent{0em} % kein Einrücken der Caption von Figures und Tabellen
\setcapwidth[c]{0.9\textwidth}
\setlength{\abovecaptionskip}{0.2cm} % Abstand der zwischen Bild- und Bildunterschrift
% Quellcode
\usepackage{listings} % für Formatierung in Quelltexten
\definecolor{grau}{gray}{0.25}
\lstset{
	extendedchars=true,
	basicstyle=\tiny\ttfamily,
	%basicstyle=\footnotesize\ttfamily,
	tabsize=2,
	keywordstyle=\textbf,
	commentstyle=\color{grau},
	stringstyle=\textit,
	numbers=left,
	numberstyle=\tiny,
	% für schönen Zeilenumbruch
	breakautoindent  = true,
	breakindent      = 2em,
	breaklines       = true,
	postbreak        = ,
	prebreak         = \raisebox{-.8ex}[0ex][0ex]{\Righttorque},
}
% linksbündige Fußboten
\deffootnote{1.5em}{1em}{\makebox[1.5em][l]{\thefootnotemark}}

\typearea{14} % typearea berechnet einen sinnvollen Satzspiegel (das heißt die Seitenränder) siehe auch http://www.ctan.org/pkg/typearea. Diese Berechnung befindet sich am Schluss, damit die Einstellungen oben berücksichtigt werden


%Überschriftennummerierung
\setcounter{secnumdepth}{3}
%Tiefe des Inhaltsverzeichnisses
\setcounter{tocdepth}{3}


\hypersetup{% 
  pdfborder= 0 0 0 
}

%lstset Formatierungen
\lstset{
  basicstyle=\fontsize{11}{13}\selectfont\ttfamily
}

%Formatierung von Abbildungs und Tabellenverzeichnis/caption
\renewcommand{\cftfigpresnum}{Abb. }
\renewcommand{\cfttabpresnum}{Tab. }

\renewcommand{\cftfigaftersnum}{:}
\renewcommand{\cfttabaftersnum}{:}

\setlength{\cftfignumwidth}{2cm}
\setlength{\cfttabnumwidth}{2cm}

\setlength{\cftfigindent}{0cm}
\setlength{\cfttabindent}{0cm}

\renewcaptionname{ngerman}{\figurename}{Abb.}
\renewcommand{\tablename}{Tab.}

%Tabellenverzeichnis mit fortlaufenden Nummern aber immer noch nach Kapiteln gruppiert. 
%\counterwithout{table}{chapter}

%Natbib zum zitieren im Stil (Autor Jahr, Seite)
\usepackage{natbib}
%Wenn man die Havard Zitierweise benutzt:
%\bibliographystyle{dcugerman}
%\usepackage{har2nat}
%\providecommand\harvardand{}
%\renewcommand{\harvardand}{und}

%Wenn man spbasic verwendet
%\bibliographystyle{spbasic} %Im spbasic.bst Dokument sind die dt. Abkürzungen die im Literaturverzeichnis erstellt werden, möchte man die Arbeit auf Englisch schreiben müssen dort die Abkürzungen angepasst werden
\bibliographystyle{extension}

\setcitestyle{aysep={},yysep={;}}

%Befehle für Multizitate
\newcommand{\doublecite}[2]{(#1; #2)}
\newcommand{\tripplecite}[3]{(#1; #2; #3)}
\newcommand{\quadrocite}[4]{(#1; #2; #3; #4)}
\newcommand{\quintcite}[5]{(#1; #2; #3; #4; #5)}

% fortlaufende nummerierung der footnotes
\usepackage{chngcntr}
\counterwithout{footnote}{chapter}

% einruecken des textes pro abschnitt
%\usepackage{parskip}
%\setlength{\parindent}{4mm} % Importiere die Einstellungen

%Versuch das Seitenlayout zu ändern. Dieses muss aber nach der Titelseite wieder geändert werden. Darum gibt es den gleichen Befehl nochmal am Ende des Dokumentes der Titelseite
\usepackage{geometry}
%\setlength{\textwidth}{14.5cm}
\setlength{\textheight}{24cm}
\setlength{\hoffset}{9mm}
%\setlength{\voffset}{-7mm}

% hier beginnt der eigentliche Inhalt
\begin{document}

\pagenumbering{roman}
%Titelseite
%entsprechend Autor, Titel, Matrikelnummer, Email, Abgabetermin, Arbeitsart eintragen
\author{Frederik Hendricks-Kühn}
\newcommand{\adress}{Universitätsstr. 31}
\newcommand{\postalcode}{44789 Bochum}
\title{Transformation eines traditionellen Unternehmens in ein agiles Unternehmen mit besonderem Fokus auf die Struktur}
\newcommand{\matrikel}{3080532}
\newcommand{\email}{frederik.hendricks-kuehn@stud.uni-due.de}
\newcommand{\termin}{06.02.2022}
\newcommand{\studiengang}{Wirtschaftsinformatik B.Sc.}
\newcommand{\arbeitstyp}{Bachelor-Seminar}
\newcommand{\gutachter}{Sarah Seufert}
\newcommand{\zweitgutachter}{}
\newcommand{\aktuellessemester}{7}
\newcommand{\fachsemester}{7}


% Titelseite
\pagestyle{empty}
\clearscrheadings \clearscrplain

%Hier wird der linke Seitenrand auf seinen Standardwert zurückgesetzt
\setlength{\hoffset}{0mm}

\begin{titlepage}
\begin{center}

%\includegraphics[scale=1.0]{grafiken/logo}

\ThisCenterWallPaper{1.0}{grafiken/iis-deckblatt}



\begin{minipage}{12cm}
\vspace{1.5cm}
\centering
\Large
\arbeitstyp \\ \vspace{1cm}
\end{minipage}
\vspace{1cm}

\begin{minipage}{12cm}
\centering
Thema:\\
\Large
\thetitle
\end{minipage}
\vspace{5cm}

\begin{minipage}{12cm}
\centering
\normalsize
Vorgelegt der Fakult\"at für Wirtschaftswissenschaften der Universit\"at Duisburg-Essen
\end{minipage}
\end{center}
\vspace{1cm}

\begin{normalsize}
\centering
von \\
\theauthor \\
\adress \\
\postalcode \\
Matrikelnummer: \matrikel \\
%E-Mail:
\href{mailto:\email}{\email}\\
\vspace{1cm}
Abgabetermin: \termin
\vspace{0.5cm}
%Erstgutachter: \gutachter \\
%Zweitgutachter: \zweitgutachter \\

%Versuch mit Tabelle
\begin{tabular}{ll}
Erstgutachter: & \gutachter\\
%Zweitgutachter: & Prof. Dr. Musterfrau\\
 &\\
Studiensemester: & \aktuellessemester \\
 &  \fachsemester. Fachsemester \\
\end{tabular}\\
\vspace{0.5cm}

%\begin{tabular}{ll}
%Studiensemester: & \aktuellessemester \\
% &  \fachsemester. Fachsemester \\
%\end{tabular}\\
%\vspace{0.5cm}

%Studiensemester: \aktuellessemester\\ 
%Fachsemester: \fachsemester





\end{normalsize}

\end{titlepage}

%\clearpage

%Layout wieder auf den 3cm Rand zurücksetzen
\setlength{\textheight}{24cm}
\setlength{\hoffset}{9mm}

\pagenumbering{Roman} % große Römische Seitenummerierung

% normale Kopf- und Fußzeilen für den Rest
\pagestyle{scrheadings}
\clearscrheadings
\automark[chapter]{chapter}
\ohead*{\pagemark}
\ihead{\headmark}



%Inhaltsverzeichnis
\tableofcontents

%Abkürzungsverzeichniss muss selbst geführt werden
%Nur tatsächlich verwendete Abkürzungen werden auch aufgeführt 
\chapter*{Abkürzungsverzeichnis}
\addcontentsline{toc}{chapter}{Abkürzungsverzeichnis}

% Alphabetische Reihenfolge!
\begin{acronym}


\acro{MA}{Mitarbeiter}
\acro{VUKA}{Volatilität, Unsicherheit, Komplexität, Ambiguität}
\acro{WI}{Wirtschaftsinformatik}

\end{acronym}

%Abbildungsverzeichnis
\listoffigures
%Tabellenverzeichnis
\listoftables
%To Do Liste kann natürlich bei Bedarf verwendet werden
\listoftodos

% Beginn des Inhalts
\pagenumbering{arabic}
\chapter{Motivation und Notwendigkeit}
\section{Motivation der Forschungsarbeit}
Traditionelle Unternehmen zeichnen sich meist dadurch aus, dass ihre Struktur stark hierarchisch aufgebaut ist. An der Spitze stehen wenige Personen, welche Entscheidungen und Anweisungen von den häheren Ebenen an die unteren durchgeben. Führungskräfte besitzen ein hohes Machtpotenzial und agieren als Erteiler von Anweisungen. Zumeist sind diese Unternehmen meist funktional aufgebaut, d.h. die einzelnen Abteilungen sind organisatorisch voneinander getrennt. Ein Wissenstransfer kann nur von oben nach unten erfolgen, jedoch nicht zwischen einzelnen Abteilungen.

Doch heutzutage ist es nicht in jedem Fall sinnvoll in solchen starren Strukturen zu verharren. Unternehmen müssen agiler sein und schneller auf neue Möglichkeiten und Anforderungen reagieren können. Damit ihnen dies gelingt, muss jedoch eine umfangreiche Transformation vorgenommen werden, welche ein traditionelles Unternehmen in ein agiles überführt. Was Unternehmen somit brauchen sind konkrete Maßnahmen, die getroffen werden können, um solch eine Transformation erfolgreich zu bewerkstelligen.


Die Forschung beschränkt sich hierbei meist auf die Beschreibung der Eigenschaften von agilen Unternehmen. Häufig werden mehrere Unternehmen und deren Vorgehensweisen in Rahmen von Fallstudien analysiert. Konkret wird betrachtet, welche Maßnahmen durchgeführt wurden, um eine strukturelle Transformation durchzuführen. 

Woran es jedoch mangelt ist eine umfassende Übersicht von Maßnahmen, welche getroffen werden können, um eine Transformation erfolgreich durchzuführen. Besonders auf der Strukturebene sind diese häufig nur erwähnt, aber nicht zwingend in eigenes Framework eingebettet. Die vorliegende Arbeit soll dabei helfen, konkrete Maßnahmen aus Fallstudien zu strukturieren und zu belegen wie durch agiles Arbeiten davon profitiert werden kann. Diese Arbeit soll somit die Frage beantworten: \glqq Welche strukturellen Maßnahmen können getroffen werden, um ein traditionelles Unternehmen in ein agiles Unternehmen zu transformieren?\grqq{}. Die ermittelten Maßnahmen werden anschließend sortiert und diejenigen, die der Strukturebene zuordenbar sind, werden anschließend in vier Kategorien verordnet.

\begin{enumerate}
    \item Maßnahmen, die sich mit der (Neu-)Gestaltung von Teams und Gruppen, welche gemeinsam im Unternehmen zusammen arbeiten (s. Abschnitt \ref{Maßnahmen: Teams}). 
    
    \item Maßnahmen, welche seitens des oberen und mittleren Managements getroffen werden müssen, um einen optimalen Arbeitsprozess zu ermöglichen (s. Abschnitt \ref{Maßnahmen: Management}).
    
    \item Maßnahmen, welche auf technischer Ebene, insbesondere auf IT-Ebene, getroffen werden müssen (s. Abschnitt \ref{Maßnahmen: Technik}).
    
    \item Maßnahmen, die sich mit der Gestaltung von Büros, Arbeitsplätzen und Arbeitsmöglichkeiten auseinandersetzen (s. Abschnitt \ref{Maßnahmen: Arbeitsplatze}).
\end{enumerate}

Anschließend wird betrachtet, in welcher Reihenfolge die notwendigen Maßnahmen idealerweise auszuführen sind. Ziel dieser Arbeit soll es somit sein, eine umfassende Übersicht über mögliche strukturelle Maßnahmen und ihren Anwendungsbereich zu erhalten. 

Der weitere Aufbau dieser Arbeit ist dabei wie folgt: Zunächst werden einzelne wichtige Begriffe hinsichtlich ihrer Bedeutung aus verschiedenen Perspektiven beleuchtet und definiert. Dazu gehört insbesondere der Begriff der Agilität (Abschn. \ref{def: Agilitaet}) und der der Digitalen Transformation (Abschn. \ref{def: DigitaleTransformation}). Dieser wird explizit von den Begriffen \textit{Digitization} (Abschn. \ref{def: Digitization}) und \textit{Digitalisierung} (Abschn. \ref{def: Digitalisierung}). abgegrenzt. Anschließend wird im dritten Kapitel die verwendete Methode vorgestellt, mit Hilfe derer Literatur gefunden und analysiert wurde. Im darauffolgenden Kapitel werden die daraus erschlossenen Maßnahmen gesammelt und in eigens erstellte Kategorien sortiert. Als abschließendes Fazit folgt ein Kapitel, welches die wichtigsten Aspekte zusammenfasst und einen Ausblick darüber gegeben, was sich aufbauend auf dieser Arbeit an weiteren Forschungsmöglichkeiten ergibt.


\section{Notwendigkeit agiler Strukturen}
Aber warum ist es für Unternehmen, heute mehr als früher, wichtig agil handeln zu können? Unternehmen sind nicht isoliert zu betrachten, sondern immer mit ihrer Umwelt in Verbindung zu setzen. Diese lässt sich durch die folgenden Begriffe beschreiben \ac{VUKA} \citep[S.4]{Dautovic.2021}. Das bedeutet, dass Unternehmen einer Umwelt gegenüber stehen, die sie selbst kaum (oder nur wenig) aktiv beeinflussen können. Neue Technologien entstehen, während alte verblassen und weniger wichtig werden. Neue Konkurrenten können die Vorteile der neuen Umwelt nutzen, um zu Wettbewerbsvorteilen zu gelangen, die es vorher nicht gab \citep[S.241]{Sambamurthy.2003}. Nischen und Marktpotenziale werden von agilen Unternehmen früher erkannt \citep[S.241]{Sambamurthy.2003}. Dadurch können etablierte Unternehmen ihre bisherige Stellung und Wettbewerbsvorteile verlieren. 

Unternehmen müssen folglich auf einen sich verändernden und dynamischen Markt reagieren können \citep[S.248]{Schweitzer.2021}. Um das bestmöglich zu erreichen, müssen Unternehmen selbst dynamisch und flexibel sein. Sie müssen Ausschau nach neuen Möglichkeiten und Innovationen halten \citep[S.242]{Sambamurthy.2003}. Das Ziel von Unternehmen ist somit entweder neu auftretende Geschäftsfelder zu erkennen und deren Chancen zu nutzen oder in neu entstehende Märkte vorzudringen. In beiden Fällen ist eine agile Ausrichtung des Unternehmens hilfreich, um diese emergenten Handlungsoptionen zu erkennen und zu nutzen. Agile Vorgehen ermöglicht eine höhere Reaktionsgeschwindigkeit auf Veränderungen \citep[S.248]{Schweitzer.2021}. Wichtige Ziele sind dabei eine bessere Kundenzufriedenheit und -orientierung zu erreichen, sowie die eigenen internen Abläufe und Geschäftsprozesse effizienter zu gestalten \citep[S.165]{vanderMeulen.2020}. 

Die Struktur eines Unternehmens bestimmt dabei in einem hohen Grad wie flexibel es reagieren kann. Ist das Unternehmen stark hierarchisch und funktional geprägt, ist es für einzelne Mitarbeiter schwieriger notwendige Änderungen oder Vorschläge anzustoßen. Die Zusammenarbeit mit anderen Unternehmensbereichen ist erschwert. Die vorgestellten Maßnahmen zielen darauf ab, bestehende Silos und Verhaltensmuster aufzubrechen, sodass ein Unternehmen schneller reagieren kann, als dies durch eine funktionale Trennung möglich wäre. Denn nur die Unternehmen, die ständig neue Innovationen hervorbringen und die Begebenheiten des Marktes erkennen und zu ihrem Vorteil nutzen, können auf Dauer erfolgreich sein \citep[S.241]{Sambamurthy.2003}









\chapter{Grundlegende Definitionen}
\section{Agilität}
\label{def: Agilitaet}

Definition von Conboy zu Flexibilität und dann anschließend auseinander nehmen und deuten




\section{Digitization, Digitalisierung und Digitale Transformation}
\label{Digitaziation, Digitalisierung und Digitale Transformation}

Im folgenden wird der Begriff der \textit{digitalen Transformation} näher betrachtet. Dies erfolgt anhand von Definitionen verschiedener Lexika und Glossare. Hierbei wird ein besonderer Fokus auf der Unterscheidung zum Begriff \textit{Digitalisierung} und \textit{Digitization}. Hierbei ist zu beachten, dass es für den Begriff der Digitization im deutschen keine eigene Übersetzung gibt. Stattdessen wird dieser gemeinsam mit dem Begriff \textit{digitisation} als Digitalisierung übersetzt. 

Jedoch ist es praktischer diese beiden Begriffe von einander getrennt zu beachten. Digitization bezeichnet die \glqq Überführung von Informationen von einer analogen in eine digitale Speicherform\grqq{} \citep{EnzyWI.Digitalisierung}. Beispiele dafür können vielfältig sein: Die Übertragung von Schalplatten und Kassetten zu MP3-Dateien, die Angebote eines Supermarktes nicht mehr auf Postern sondern auf Bildschirmen am Eingang des Marktes, etc... Hierdurch alleine ergibt sich noch kein Mehrwert für ein Unternehmen. 

Dieser Mehrwert kann nämlich erst dann entstehen, wenn auch einzelne Aufgaben und Prozesse an die IT ausgelagert werden. Eine Möglichkeit hierbei wäre die (Teil-) Automatisierung einzelner Aufgaben simpler Natur \citep{EnzyWI.Digitalisierung}. Mittlerweile ist jedoch auch immer weiter die Automatisierung komplexerer und unstrukturierter Aufgaben zu beobachten. Die Anzahl Industrieroboter, welche als Indikator für automatisierte Aufgaben genutzt werden können, steigt rasant an. Waren es 2010 noch ca. 1,05 Millionen, hat sich diese Zahl bis 2020 auf ca. 3,01 Millionen nahezu verdreifacht \citep{Statistak.Industrieroboter}. Die Digitalisierung sorgt also nicht nur für die Übertragung von Daten, sondern auch für die Automatisierung von Geschäftsprozessen und der Entstehung neuer innovativer Ideen. 

Der Begriff der digitalen Transformation geht dabei noch weiter. Hier liegt der Fokus auf der Veränderung ganzer Geschäftsmodelle und dem hervor bringen neuer Marktpotenziale. Geschäftsmodelle von Facebook, Google, Apple und Co wären ohne die vorangegangen Schritte nicht denkbar. Die digitale Transformation erfasst ein Unternehmen somit nicht nur im Kern, sondern auch den gesamten Kontext in dem es sich befindet und folglich auch das alltägliche Leben \citep{EnzyWI.DigitaleTransformation}. Ziel einer solchen Transformation ist es ein Unternehmen zukunftsicherer zu gestalten und neuen Herausforderungen gegeüber bereit zu sein. 





















\section{Organisatzionsstruktur}

%Zu dem Begriff der Agilität können in der Literatur diverse Definitionen gefunden werden. Die meisten stammen dabei aus der Sotware- oder Produktentwicklung.  


\chapter{Methode}

In der vorliegenden Seminararbeit wurde eine Literaturrecherche durchgeführt, um einen Überblick über bereits etablierte Maßnahmen zu erhalten. Hierbei wurde der Fokus auf die Analyse von Fallstudien gelegt, da in diesen häufig mehrere Unternehmen untersucht werden und welche Maßnahmen diese ?? haben um sich erfolgreicher zu positionieren. Hierbei wurde zunächst die Datenbank Aisel.aisnet.org verwendet.

Da in der ersten Recherche nur qualitative und peer-geprüfte Fallstudien gewünscht waren wurde beschlossen sich zunächst auf Journal Artikel zu beschränken. Ein weiterer wichtiger Filter, der hier angewandt wurde, war das nur ?? ab dem Jahre 2000 betrachtet werden. Damit sollte vermieden werden, dass veraltete Maßnahmen (wie bspw. die Anschaffung von Desktop-PCs) identifiziert werden. Der Fokus lag somit auf aktuellen, erfolgreichen Maßnahmen, die die Struktur eines Unternehmens beeinflussen. Die Stichwortsuche wurde dafür mit dem folgenden Suchbegriff gestartet:

\verb+transformation AND agil AND structure AND hierarchy AND actions+. Damit sollte sichergestellt werden, dass möglichst viele Ergebnisse Maßnahmen zu dem gewünschten Themengebiet aufweisen. 

Diese erste Suche ergab 88 Paper und Berichte. Diese wurden anschließend nach ihrem Titel gefiltert. Paper, deren Titel keinen direkten Bezug zum Thema aufwiesen wurden direkt aussortiert. 19 der 88 Paper hatten dabei Titel, welche sich vielversprechend anhörten, sodass bei diesen nun das Abstract und ggf. die Einleitung gelesen wurde, um die tatsächliche Eignung zu prüfen. Dieses finale Filterkriterium ergab dann 7 Paper, welche analysiert wurden und eine erste Basis für Maßnahmen lieferten. 

Hierbei sei angemerkt, dass aufgrund der Aktualität mancher Artikel bei diesen das Kriterium \textit{Backward search} angewandt wurde, um vorherige Forschungsergebnisse nicht zu vernachlässigen. Ebenfalls wurden in der ersten Literatursuche, agile Softwareentwicklungsmethoden und -leitsätze analysiert, um ein breiteres Spektrum an Maßnahmen zu erhalten. Hierbei sei insbesondere das agile Manifest \cite{agile.Manifesto} und der offizielle Scrum Guide \cite{scrumGuide} genannt.




\chapter{Maßnahmen}




\section{Maßnahmen zur Gestaltung von Teams}
\label{Maßnahmen: Teams}

\section{Maßnahmen zur Neuausrichtung des Managements}
\label{Maßnahmen: Management}

\section{Maßnahmen auf Technischer und IT-Ebene}
\label{Maßnahmen: Technik}

\section{Maßnahmen zur Gestaltung von Arbeitsplätzen}
\label{Maßnahmen: Arbeitsplatze}







\section{Zusammenfassung und Priorisierung}
\include{tex/05_Fazit}
\include{tex/06_anhang}
\chapter*{Eidesstattliche Erklärung}
Ich versichere an Eides statt durch meine Unterschrift, dass ich die vorstehende Arbeit selbständig und ohne fremde Hilfe angefertigt und alle Stellen, die ich wörtlich oder annähernd wörtlich aus Veröffentlichungen entnommen habe, als solche kenntlich gemacht habe, mich auch keiner anderen als der angegebenen Literatur oder sonstiger Hilfsmittel bedient habe. Die Arbeit hat in dieser oder ähnlicher Form noch keiner anderen Prüfungsbehörde vorgelegen.\\

\noindent\rule{5cm}{1pt} \hspace{2cm} \noindent\rule{5cm}{1pt}\\
\vspace{3cm}
Ort, Datum \hspace{5cm} Unterschrift\\

%Literaturverzeichnis
\bibliography{bibliographie}

%Anhang
\include{tex/anhang}

%Eigenständigskeitserklärung
\include{tex/erklaerung}

\end{document}