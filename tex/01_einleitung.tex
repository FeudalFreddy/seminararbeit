\chapter{Motivation und Notwendigkeit}
\section{Motivation der Forschungsarbeit}
Traditionelle Unternehmen zeichnen sich meist dadurch aus, dass ihre Struktur stark hierarchisch aufgebaut ist. An der Spitze stehen wenige Personen, welche Entscheidungen und Anweisungen von den häheren Ebenen an die unteren durchgeben. Führungskräfte besitzen ein hohes Machtpotenzial und agieren als Erteiler von Anweisungen. Zumeist sind diese Unternehmen meist funktional aufgebaut, d.h. die einzelnen Abteilungen sind organisatorisch voneinander getrennt. Ein Wissenstransfer kann nur von oben nach unten erfolgen, jedoch nicht zwischen einzelnen Abteilungen.

Doch heutzutage ist es nicht in jedem Fall sinnvoll in solchen starren Strukturen zu verharren. Unternehmen müssen agiler sein und schneller auf neue Möglichkeiten und Anforderungen reagieren können. Damit ihnen dies gelingt, muss jedoch eine umfangreiche Transformation vorgenommen werden, welche ein traditionelles Unternehmen in ein agiles überführt. Was Unternehmen somit brauchen sind konkrete Maßnahmen, die getroffen werden können, um solch eine Transformation erfolgreich zu bewerkstelligen.


Die Forschung beschränkt sich hierbei meist auf die Beschreibung der Eigenschaften von agilen Unternehmen. Häufig werden mehrere Unternehmen und deren Vorgehensweisen in Rahmen von Fallstudien analysiert. Konkret wird betrachtet, welche Maßnahmen durchgeführt wurden, um eine strukturelle Transformation durchzuführen. 

Woran es jedoch mangelt ist eine umfassende Übersicht von Maßnahmen, welche getroffen werden können, um eine Transformation erfolgreich durchzuführen. Besonders auf der Strukturebene sind diese häufig nur erwähnt, aber nicht zwingend in eigenes Framework eingebettet. Die vorliegende Arbeit soll dabei helfen, konkrete Maßnahmen aus Fallstudien zu strukturieren und zu belegen wie durch agiles Arbeiten davon profitiert werden kann. Diese Arbeit soll somit die Frage beantworten: \glqq Welche strukturellen Maßnahmen können getroffen werden, um ein traditionelles Unternehmen in ein agiles Unternehmen zu transformieren?\grqq{}. Die ermittelten Maßnahmen werden anschließend sortiert und diejenigen, die der Strukturebene zuordenbar sind, werden anschließend in vier Kategorien verordnet.

\begin{enumerate}
    \item Maßnahmen, die sich mit der (Neu-)Gestaltung von Teams und Gruppen, welche gemeinsam im Unternehmen zusammen arbeiten (s. Abschnitt \ref{Maßnahmen: Teams}). 
    
    \item Maßnahmen, welche seitens des oberen und mittleren Managements getroffen werden müssen, um einen optimalen Arbeitsprozess zu ermöglichen (s. Abschnitt \ref{Maßnahmen: Management}).
    
    \item Maßnahmen, welche auf technischer Ebene, insbesondere auf IT-Ebene, getroffen werden müssen (s. Abschnitt \ref{Maßnahmen: Technik}).
    
    \item Maßnahmen, die sich mit der Gestaltung von Büros, Arbeitsplätzen und Arbeitsmöglichkeiten auseinandersetzen (s. Abschnitt \ref{Maßnahmen: Arbeitsplatze}).
\end{enumerate}

Anschließend wird betrachtet, in welcher Reihenfolge die notwendigen Maßnahmen idealerweise auszuführen sind. Ziel dieser Arbeit soll es somit sein, eine umfassende Übersicht über mögliche strukturelle Maßnahmen und ihren Anwendungsbereich zu erhalten. 

Der weitere Aufbau dieser Arbeit ist dabei wie folgt: Zunächst werden einzelne wichtige Begriffe hinsichtlich ihrer Bedeutung aus verschiedenen Perspektiven beleuchtet und definiert. Dazu gehört insbesondere der Begriff der Agilität (Abschn. \ref{def: Agilitaet}) und der der Digitalen Transformation (Abschn. \ref{def: DigitaleTransformation}). Dieser wird explizit von den Begriffen \textit{Digitization} (Abschn. \ref{def: Digitization}) und \textit{Digitalisierung} (Abschn. \ref{def: Digitalisierung}). abgegrenzt. Anschließend wird im dritten Kapitel die verwendete Methode vorgestellt, mit Hilfe derer Literatur gefunden und analysiert wurde. Im darauffolgenden Kapitel werden die daraus erschlossenen Maßnahmen gesammelt und in eigens erstellte Kategorien sortiert. Als abschließendes Fazit folgt ein Kapitel, welches die wichtigsten Aspekte zusammenfasst und einen Ausblick darüber gegeben, was sich aufbauend auf dieser Arbeit an weiteren Forschungsmöglichkeiten ergibt.


\section{Notwendigkeit agiler Strukturen}
Aber warum ist es für Unternehmen, heute mehr als früher, wichtig agil handeln zu können? Unternehmen sind nicht isoliert zu betrachten, sondern immer mit ihrer Umwelt in Verbindung zu setzen. Diese lässt sich durch die folgenden Begriffe beschreiben \ac{VUKA} \citep[S.4]{Dautovic.2021}. Das bedeutet, dass Unternehmen einer Umwelt gegenüber stehen, die sie selbst kaum (oder nur wenig) aktiv beeinflussen können. Neue Technologien entstehen, während alte verblassen und weniger wichtig werden. Neue Konkurrenten können die Vorteile der neuen Umwelt nutzen, um zu Wettbewerbsvorteilen zu gelangen, die es vorher nicht gab \citep[S.241]{Sambamurthy.2003}. Nischen und Marktpotenziale werden von agilen Unternehmen früher erkannt \citep[S.241]{Sambamurthy.2003}. Dadurch können etablierte Unternehmen ihre bisherige Stellung und Wettbewerbsvorteile verlieren. 

Unternehmen müssen folglich auf einen sich verändernden und dynamischen Markt reagieren können \citep[S.248]{Schweitzer.2021}. Um das bestmöglich zu erreichen, müssen Unternehmen selbst dynamisch und flexibel sein. Sie müssen Ausschau nach neuen Möglichkeiten und Innovationen halten \citep[S.242]{Sambamurthy.2003}. Das Ziel von Unternehmen ist somit entweder neu auftretende Geschäftsfelder zu erkennen und deren Chancen zu nutzen oder in neu entstehende Märkte vorzudringen. In beiden Fällen ist eine agile Ausrichtung des Unternehmens hilfreich, um diese emergenten Handlungsoptionen zu erkennen und zu nutzen. Agile Vorgehen ermöglicht eine höhere Reaktionsgeschwindigkeit auf Veränderungen \citep[S.248]{Schweitzer.2021}. Wichtige Ziele sind dabei eine bessere Kundenzufriedenheit und -orientierung zu erreichen, sowie die eigenen internen Abläufe und Geschäftsprozesse effizienter zu gestalten \citep[S.165]{vanderMeulen.2020}. 

Die Struktur eines Unternehmens bestimmt dabei in einem hohen Grad wie flexibel es reagieren kann. Ist das Unternehmen stark hierarchisch und funktional geprägt, ist es für einzelne Mitarbeiter schwieriger notwendige Änderungen oder Vorschläge anzustoßen. Die Zusammenarbeit mit anderen Unternehmensbereichen ist erschwert. Die vorgestellten Maßnahmen zielen darauf ab, bestehende Silos und Verhaltensmuster aufzubrechen, sodass ein Unternehmen schneller reagieren kann, als dies durch eine funktionale Trennung möglich wäre. Denn nur die Unternehmen, die ständig neue Innovationen hervorbringen und die Begebenheiten des Marktes erkennen und zu ihrem Vorteil nutzen, können auf Dauer erfolgreich sein \citep[S.241]{Sambamurthy.2003}








