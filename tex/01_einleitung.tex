\chapter{Motivation und Notwendigkeit}

Traditionelle Unternehmen zeichnen sich meist dadurch aus, dass ihre Struktur stark hierarchisch aufgebaut ist.                 
An der Spitze steht eine geringe Anzahl an Personen, welche nahezu die vollständige Macht im Unternehmen haben und ausführen.   
Sie sorgen dafür, dass Entscheidungen und Anweisungen von den höheren Ebenen an die unteren durchgegeben werden.                
Führungskräfte besitzen folglich ein hohes Machtpotenzial und agieren als Erteiler von Anweisungen.                             


Diese Struktur bietet den Unternehmen dabei diverse Vorteile, wie Stabilität und klare Verantwortlichkeiten.
Jedoch hat sich in den letzten Jahrzehnten gezeigt, dass die Umwelt in der Unternehmen agieren sich stärker wandelt \citep[S.4]{Dautovic.2021}. 
\todo{Tolle Studie einfügen Bspw. Unternehmensneugründungen und neu Techmologiemm}
Dies hat zur Folge, dass Unternehmen einer Umwelt gegenüber stehen, die sie selbst kaum (oder nur wenig) aktiv beeinflussen können. 
Neue Technologien entstehen, während alte verblassen und weniger wichtig werden.
So können neue Konkurrenten entstehen, die diese Wettbewerbsvorteile schneller erkennen und zu ihrem Vorteil nutzen \citep[S.241]{Sambamurthy.2003}. 
Es ist möglich, dass etablierte Unternehmen ihre bisherige Marktposition verlieren, da andere Unternehmen schneller reagieren können.


Ein wichtiger Faktor, der Unternehmen dabei hilft flexibler und schneller zu reagieren ist dabei die Agilität eines Unternehmens \citep[S.241]{Sambamurthy.2003}. 
Die Struktur eines Unternehmens bestimmt dabei in einem hohen Grad wie flexibel es reagieren kann. 
Ist das Unternehmen stark hierarchisch und funktional geprägt, ist es für einzelne Mitarbeiter schwieriger notwendige Änderungen oder Vorschläge anzustoßen.
Ein agiles Unternehmen kann schnellere Entscheidungen treffen, sowie schneller auf Neuerungen und Anpassungen im Markt reagieren \citep[S.248]{Schweitzer.2021}.
Wichtige Ziele in der Agilität sind eine hohe Kundenzufriedenheit und -orientierung, sowie die Verbesserung der eigenen internen Abläufe und Geschäftsprozesse \citep[S.165]{vanderMeulen.2020}. 
Ein weiterer wichtiger Aspekt ist, dass Unternehmen meist nur dann erfolgreich sind, wenn sie häufig neue Innovationen hervorbringen und die Begebenheiten des Marktes erkennen und zu ihren Vorteil nutzen \citep[S.241+242]{Sambamurthy.2003}
Agile Vorgehen ermöglichen eine höhere Reaktionsgeschwindigkeit auf Veränderungen \citep[S.248]{Schweitzer.2021}.
Dies zeigt, dass ein agiles Vorgehen Wettberwerbsvorteile ermöglichen kann.
Um ein traditionelles Unternehmen in ein agiles zu überführen ist jedoch eine umfassende Transformation notwendig.


Es gibt dabei diverse Forschungsarbeiten, die sich mit strukturellen Maßnahmen auseinandersetzen, welche ein traditionelles Unternehmen in ein agiles überführen können.
\todo{Literatur aufzählen und erläutern}
Diese Seminararbeit soll eine Liste der relevantesten Maßnahmen abbilden und diese in den Kontext einer agilen Transformation einordnen.
Die vorliegende Arbeit soll somit dabei helfen, erfolgreiche Maßnahmen aus der bisherigen Forschung zu strukturieren.
Es soll die folgenden Frage beantwortet werden: \glqq Welche strukturellen Maßnahmen können getroffen werden, um ein traditionelles Unternehmen in ein agiles Unternehmen zu transformieren?\grqq{}. 


Der weitere Aufbau dieser Arbeit ist dabei wie folgt: 
Zunächst werden einzelne wichtige Begriffe hinsichtlich ihrer Bedeutung aus verschiedenen Perspektiven beleuchtet und definiert. 
Dazu gehört insbesondere der Begriff der Agilität (Abschn. \ref{def: Agilitaet}) und der der Digitalen Transformation (Abschn. \ref{def: DigitaleTransformation}).
Dieser wird explizit von den Begriffen \textit{Digitization} und \textit{Digitalisierung} abgegrenzt. 
Anschließend wird im dritten Kapitel die verwendete Methode vorgestellt, mit Hilfe derer Literatur gefunden und analysiert wurde. 
Im darauffolgenden Kapitel werden erschlossenen Maßnahmen gesammelt und in eigens erstellte Kategorien sortiert. 
Als abschließendes Fazit folgt ein Kapitel, welches die wichtigsten Aspekte zusammenfasst und einen Ausblick darüber gibt, was an an weiter Forschung möglich ist.


% Wohin damit?
%Die vorgestellten Maßnahmen zielen darauf ab, bestehende Silos und Verhaltensmuster aufzubrechen, sodass ein Unternehmen schneller reagieren kann, als dies durch eine funktionale Trennung möglich wäre. 
%Darüber hinaus ist die Zusammenarbeit mit anderen Unternehmensbereichen ist erschwert. 
%Silos sind gebildet und es wird schwierig diese wieder aufzubrechen.
%Ein Wissenstransfer oder Austausch ist somit meist nur innerhalb der eigenen Abteilung möglich, jedoch nicht zwischen den einzelnen Abteilungen.
