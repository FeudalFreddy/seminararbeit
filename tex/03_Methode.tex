\chapter{Methode}

In der vorliegenden Seminararbeit wurde eine Literaturrecherche durchgeführt, um einen Überblick über bereits etablierte Maßnahmen zu erhalten. Hierbei wurde der Fokus auf die Analyse von Fallstudien gelegt, da in diesen häufig mehrere Unternehmen untersucht werden und welche Maßnahmen diese ?? haben um sich erfolgreicher zu positionieren. Hierbei wurde zunächst die Datenbank Aisel.aisnet.org verwendet.

Da in der ersten Recherche nur qualitative und peer-geprüfte Fallstudien gewünscht waren wurde beschlossen sich zunächst auf Journal Artikel zu beschränken. Ein weiterer wichtiger Filter, der hier angewandt wurde, war das nur ?? ab dem Jahre 2000 betrachtet werden. Damit sollte vermieden werden, dass veraltete Maßnahmen (wie bspw. die Anschaffung von Desktop-PCs) identifiziert werden. Der Fokus lag somit auf aktuellen, erfolgreichen Maßnahmen, die die Struktur eines Unternehmens beeinflussen. Die Stichwortsuche wurde dafür mit dem folgenden Suchbegriff gestartet:

\verb+transformation AND agil AND structure AND hierarchy AND actions+. Damit sollte sichergestellt werden, dass möglichst viele Ergebnisse Maßnahmen zu dem gewünschten Themengebiet aufweisen. 

Diese erste Suche ergab 88 Paper und Berichte. Diese wurden anschließend nach ihrem Titel gefiltert. Paper, deren Titel keinen direkten Bezug zum Thema aufwiesen wurden direkt aussortiert. 19 der 88 Paper hatten dabei Titel, welche sich vielversprechend anhörten, sodass bei diesen nun das Abstract und ggf. die Einleitung gelesen wurde, um die tatsächliche Eignung zu prüfen. Dieses finale Filterkriterium ergab dann 7 Paper, welche analysiert wurden und eine erste Basis für Maßnahmen lieferten. 

Hierbei sei angemerkt, dass aufgrund der Aktualität mancher Artikel bei diesen das Kriterium \textit{Backward search} angewandt wurde, um vorherige Forschungsergebnisse nicht zu vernachlässigen. Ebenfalls wurden in der ersten Literatursuche, agile Softwareentwicklungsmethoden und -leitsätze analysiert, um ein breiteres Spektrum an Maßnahmen zu erhalten. Hierbei sei insbesondere das agile Manifest \cite{agile.Manifesto} und der offizielle Scrum Guide \cite{scrumGuide} genannt.



