\chapter{Maßnahmen}


%Die ermittelten Maßnahmen werden anschließend sortiert und diejenigen, die der Strukturebene zuordenbar sind, werden anschließend in vier Kategorien verordnet.

%\begin{enumerate}
%    \item Maßnahmen, die sich mit der (Neu-)Gestaltung von Teams und Gruppen, welche gemeinsam im Unternehmen zusammen arbeiten (s. Abschnitt \ref{Maßnahmen: Teams}). 
    
%    \item Maßnahmen, welche seitens des oberen und mittleren Managements getroffen werden müssen, um einen optimalen Arbeitsprozess zu ermöglichen (s. Abschnitt \ref{Maßnahmen: Management}).
    
%    \item Maßnahmen, welche auf technischer Ebene, insbesondere auf IT-Ebene, getroffen werden müssen (s. Abschnitt \ref{Maßnahmen: Technik}).
    
%    \item Maßnahmen, die sich mit der Gestaltung von Büros, Arbeitsplätzen und Arbeitsmöglichkeiten auseinandersetzen (s. Abschnitt \ref{Maßnahmen: Arbeitsplatze}).
%\end{enumerate}

%Anschließend wird betrachtet, in welcher Reihenfolge die notwendigen Maßnahmen idealerweise auszuführen sind. 
%Ziel dieser Arbeit soll es somit sein, eine umfassende Übersicht über mögliche strukturelle Maßnahmen und ihren Anwendungsbereich zu erhalten. 


\section{Übersicht der gefundenen Maßnahmen}
\label{Maßnahmen}

Im folgenden werden die gefunden Maßnahmen zunächst gelistet und anschließend verschiedenen Kategorien zugeordnet und gedeutet.










\section{Maßnahmen zur Gestaltung von Teams}
\label{Maßnahmen: Teams}

\section{Maßnahmen zur Neuausrichtung des Managements}
\label{Maßnahmen: Management}

\section{Maßnahmen auf Technischer und IT-Ebene}
\label{Maßnahmen: Technik}

\section{Maßnahmen zur Gestaltung von Arbeitsplätzen}
\label{Maßnahmen: Arbeitsplatze}







\section{Zusammenfassung und Priorisierung}