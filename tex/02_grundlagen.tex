\chapter{Grundlegende Definitionen}
\section{Agilität}
\label{def: Agilitaet}

Definition von Conboy zu Flexibilität und dann anschließend auseinander nehmen und deuten




\section{Digitization, Digitalisierung und Digitale Transformation}
\label{Digitaziation, Digitalisierung und Digitale Transformation}

Im folgenden wird der Begriff der \textit{digitalen Transformation} näher betrachtet. Dies erfolgt anhand von Definitionen verschiedener Lexika und Glossare. Hierbei wird ein besonderer Fokus auf der Unterscheidung zum Begriff \textit{Digitalisierung} und \textit{Digitization}. Hierbei ist zu beachten, dass es für den Begriff der Digitization im deutschen keine eigene Übersetzung gibt. Stattdessen wird dieser gemeinsam mit dem Begriff \textit{digitisation} als Digitalisierung übersetzt. 

Jedoch ist es praktischer diese beiden Begriffe von einander getrennt zu beachten. Digitization bezeichnet die \glqq Überführung von Informationen von einer analogen in eine digitale Speicherform\grqq{} \citep{EnzyWI.Digitalisierung}. Beispiele dafür können vielfältig sein: Die Übertragung von Schalplatten und Kassetten zu MP3-Dateien, die Angebote eines Supermarktes nicht mehr auf Postern sondern auf Bildschirmen am Eingang des Marktes, etc... Hierdurch alleine ergibt sich noch kein Mehrwert für ein Unternehmen. 

Dieser Mehrwert kann nämlich erst dann entstehen, wenn auch einzelne Aufgaben und Prozesse an die IT ausgelagert werden. Eine Möglichkeit hierbei wäre die (Teil-) Automatisierung einzelner Aufgaben simpler Natur \citep{EnzyWI.Digitalisierung}. Mittlerweile ist jedoch auch immer weiter die Automatisierung komplexerer und unstrukturierter Aufgaben zu beobachten. Die Anzahl Industrieroboter, welche als Indikator für automatisierte Aufgaben genutzt werden können, steigt rasant an. Waren es 2010 noch ca. 1,05 Millionen, hat sich diese Zahl bis 2020 auf ca. 3,01 Millionen nahezu verdreifacht \citep{Statistak.Industrieroboter}. Die Digitalisierung sorgt also nicht nur für die Übertragung von Daten, sondern auch für die Automatisierung von Geschäftsprozessen und der Entstehung neuer innovativer Ideen. 

Der Begriff der digitalen Transformation geht dabei noch weiter. Hier liegt der Fokus auf der Veränderung ganzer Geschäftsmodelle und dem hervor bringen neuer Marktpotenziale. Geschäftsmodelle von Facebook, Google, Apple und Co wären ohne die vorangegangen Schritte nicht denkbar. Die digitale Transformation erfasst ein Unternehmen somit nicht nur im Kern, sondern auch den gesamten Kontext in dem es sich befindet und folglich auch das alltägliche Leben \citep{EnzyWI.DigitaleTransformation}. Ziel einer solchen Transformation ist es ein Unternehmen zukunftsicherer zu gestalten und neuen Herausforderungen gegeüber bereit zu sein. 





















\section{Organisatzionsstruktur}

%Zu dem Begriff der Agilität können in der Literatur diverse Definitionen gefunden werden. Die meisten stammen dabei aus der Sotware- oder Produktentwicklung.  

